%!TEX root = ../thesis.tex
\chapter*{概要}
\thispagestyle{empty}
%
\begin{center}
  \scalebox{1.5}{カメラ画像と目標方向を用いたend-end学習による}\\
  \scalebox{1.5}{分岐路でのルート選択可能なのnavigation手法の提案}\\
\end{center}
\vspace{1.0zh}
%

近年,カメラ画像に基づいた自律移動の研究が行われている.
本研究室でも,LiDAR を用いた自律移動システムの出力を教師信号として与えることで
ロボットの経路追従行動をオンラインで模倣する手法を提案し,
実験により一定の経路を周回可能であると示されている.
本研究では,従来手法をベースに,目標とする進行方向("目標方向")をデータセットと学習器の入力へ加えることで,
分岐路で「直進」と「左折」などの経路を選択可能とする機能の追加を提案する.
提案手法では,LiDAR を用いた自律移動システムの出力をカメラ画像と目標方向を用いて模倣学習し,
学習後,カメラ画像と目標方向に基づいて経路を選択可能な自律走行を行う.
また,シミュレータを用いた実験により,提案手法の有効性を検証した. 

キーワード: end-to-end学習, Navigation, 目標方向
%
\newpage
%%
\chapter*{abstract}
\thispagestyle{empty}
%
\begin{center}
  \scalebox{1.3}{An End-to-End Navigation Method for Route Selection on Branch Roads}
  \scalebox{1.3}{Using Camera Images and Target Directions}\\
\end{center}
\vspace{1.0zh}
%

In this paper, 

keywords: End-to-end Learning, Navigation, Target Direction
